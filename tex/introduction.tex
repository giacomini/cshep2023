% !TeX root = cshep2023.tex

\section*{Introduction}

\begin{frame}{Course timetable and communication}
  \begin{itemize}
  \item Usually Monday 9--11

    \begin{itemize}
    \item Changes are possible, especially with the Thursday 9--11 slot (Module 5 by Andrea Chierici)
    \item They will be communicated in due time
    \end{itemize}
  \item For any question, comment, clarification, \ldots just contact me via e-mail or Teams
  \item Are you interested in a chat on Teams or other tool for a more agile
    communication among us?
  \end{itemize}
\end{frame}

\begin{frame}{Supporting material}

  \begin{itemize}

  \item The source of this presentation and code examples are kept at
    \url{https://github.com/giacomini/cshep2023}

  \item The pdf of this and other presentations will be made available on
    Virtuale

  \item Suggested books for \Cpp{}
    \begin{itemize}
    \item B.~Stroustrup, \href{https://stroustrup.com/tour3.html}{\textit{A tour
          of C++}}, 3rd edition (available from October 2022), Addison-Wesley. A
      subset of the contents of the 2nd edition is available online at
      \url{https://isocpp.org/tour}
  
    \item B.~Stroustrup,
      \href{https://stroustrup.com/programming.html}{\textit{Programming:
          Principles and Practice Using C++}}, 2nd edition, Addison-Wesley

    \item B.~Stroustrup, \href{https://stroustrup.com/4th.html}{\textit{The C++
          Programming Language}}, $4^{th}$ edition, Addison-Wesley
    \end{itemize}

  \item Suggested online resources
    \begin{itemize}
    \item \textit{Learn \Cpp{}} \url{https://learncpp.com/}: tutorials,
      examples, exercises
    \item \textit{\Cpp{} reference} \url{https://cppreference.com/}
    \end{itemize}

  \item Other resources will be provided during the course
  \end{itemize}

\end{frame}

\begin{frame}{Exam}

  The exam consists of two parts:

  \begin{enumerate}
  \item Development of a parallel program in C++ to be prepared and executed in
    a virtual machine instantiated on the Cloud
  \item An oral part discussing the project and the concepts and technologies
    that are the subject of the course
  \end{enumerate}

  Details will follow.

\end{frame}

\begin{frame}{Platforms and tools}
  \begin{itemize}[<+->]
  \item The reference platform is Linux (Ubuntu 22.04) with the \code{gcc}
    compiler suite
  \item But any recent Linux platform is fine
  \item I can give some help to install and configure:
    \begin{itemize}[<.->]
    \item Windows: Ubuntu inside Window Subsystem for Linux
    \item macOS: XCode (-tools) with \code{gcc}
    \end{itemize}
  \item Any text editor is fine: nano, vi, emacs, gedit, geany, notepad, \ldots
  \item You can also use an Integrated Development Environment (IDE): \textbf{VS
      Code}, Visual Studio, XCode, KDevelop, Eclipse, CLion, \ldots
  \item You can experiment with online compilers
    \begin{itemize}[<.->]
    \item \url{https://godbolt.org/}
    \item \url{https://coliru.stacked-crooked.com/}
    \end{itemize}
  \end{itemize}
\end{frame}

\begin{frame}{Course description}

  \begin{itemize}
  \item Elements of computer architecture and operating systems
  \item Application of \Cpp{} programming techniques to scientific software
    development, including data abstraction, polymorphism, generic programming,
    concurrency and parallelism
  \item Use of modern \Cpp{} to safely and efficiently exploit the memory
    hierarchy and the heterogeneous nature of current computer architectures
  \item Introduction to elements of software engineering and use of effective
    development tools
  \end{itemize}

\end{frame}

\begin{frame}{Anticipation of Module 5}

  \begin{itemize}
  \item Computer hardware and technologies for the data center
  \item HTC and HPC + practice
  \item Cloud computing + practice
  \item Containers + practice
  \item Introduction to big data
  \end{itemize}

\end{frame}
